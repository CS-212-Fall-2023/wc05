\documentclass[a4paper]{exam}

\usepackage{amsmath,amssymb, amsthm}
\usepackage[a4paper]{geometry}
\usepackage{hyperref}
\usepackage{mdframed}

\title{Weekly Challenge 05: Regularity}
\author{CS 212 Nature of Computation\\Habib University}
\date{Fall 2023}

\theoremstyle{definition}
\newtheorem{definition}{Definition}

\theoremstyle{claim}
\newtheorem{claim}{Claim}

\qformat{{\large\bf \thequestion. \thequestiontitle}\hfill}
\boxedpoints

\usepackage{draftwatermark}
\SetWatermarkText{Sample Solution}
\SetWatermarkScale{3}
\printanswers

\begin{document}
\maketitle

\begin{questions}

\titledquestion{Regular?}

  Prove or disprove the following claim.
  \begin{claim}
    The language, $L = \{ w^iw^j \mid w\in\{0,1\}^*, 0 < i \le j \}$, is regular.
  \end{claim}

  \begin{solution}
    The language can be written more simply as $L = \{ w^i \mid w\in\{0,1\}^*, i \ge 2 \}$.
    \begin{proof} We prove that the language is non-regular by using the pumping lemma.

      Assume that $L$ is regular with a pumping length, $p$.\\
      Consider $s = 1^p01^p0$.\\
      Then $s\in L$ and $|s| > p$, so the pumping lemma can be applied on $s$.\\
      Decomposing $s$ into $x, y,$ and $z$ as per the pumping lemma, $y=1^m$ where $1\le m \le p$.\\
      Then $xy^0z = 1^{p-m}01^p0 \not\in L.\qquad \textcolor{red}{\bot}$
    \end{proof}
  \end{solution}
  
\end{questions}
\end{document}

%%% Local Variables:
%%% mode: latex
%%% TeX-master: t
%%% End:
